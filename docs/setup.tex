%Copyright (c) 2015 Alan Chandler, all rights reserved
%
%This file is part of PASv5, an implementation of the Patient Administration
%System used to support Accuvision's Laser Eye Clinics.
%
%PASv5 is licenced to Accuvision (and its successors in interest) free of royality payments
%and in perpetuity in the hope that it will be useful, but WITHOUT ANY WARRANTY; without even the
%implied warranty of MERCHANTABILITY or FITNESS FOR A PARTICULAR PURPOSE. Accuvision
%may modify, or employ an outside party to modify, any of the software provided that
%this modified software is only used as part of Accuvision's internal business processes.
%
%The software may be run on either Accuvision's own computers or on external computing
%facilities provided by a third party, provided that the software remains soley for use
%by Accuvision (or by potential or existing customers in interacting with Accuvision).

\documentclass[Draft]{akc}
\author{Alan Chandler}
\title{Project Manual}
\project{PI Audio Recorder}
\date{26 March 2022}
\usepackage{listings}
\lstdefinestyle{bstyle}{
  belowcaptionskip=1\baselineskip,
  breaklines=true,
  frame=L,
  columns=flexible,
  xleftmargin=\parindent,
  language=bash,
  showstringspaces=false,
  basicstyle=\footnotesize\ttfamily,
  keywordstyle=\bfseries\color{green!40!black},
  commentstyle=\itshape\color{purple!40!black},
  identifierstyle=\color{blue},
  stringstyle=\color{orange},
}
\lstset{style=bstyle}
\begin{document}
\maketitle
\abstract{%
This is a project to create a portable audio recorder.  It consists of a Raspberry PI 4 running a an nginx web server and a nodejs api server which 
provides a web interface to control recording and to spawn the ffmpeg application to record the audio, but also a data stream of the loudness
of the audio signal which is used by the client display a real time loudness graph. 
}
\tableofcontents
\section{Introduction}

This is a project to provide an audio recorder supporting the Scarlett 2 Microphone Hub and the Blue Yetti USB microphone to record the audio
signal as high quality audio. These files may later be retrieved to be used in input to other mulitmedia programs.

The core of the software runs on a raspberry pi and this makes it a portable solution which can be taken anywhere.

\section{The Hardware}

The hardware needed for the Recorder Server is :-

\begin{itemize}
\item Raspberry Pi 4
\item FLIRC Raspberry Pi Case (recommended for its thermal properties, although any case will work)
\item Raspberry Pi Universal Power Supply (5v, 2A, USBc)
\item 64GB Micro SD (SanDisk class @10 recommended) mainly for recording storage, so any size above 8G is satisfactory if you are happy to restrict recording space.
\end{itemize}

\subsection{Putting an Operating System and other Software on the Hardware}

The hardware boots and runs the software installed on a MicroSD card placed in the slot in the case
for that purpose.  The first partition of the MicroSD card \emph{must} be of type FAT32 and contain
the boot image and information associated with booting the system.  The other partitions make up the
running system.

You start with a blank card and add information to it.

In essence there are two ways to create the SD card.

\begin{enumerate}
\item From scratch, creating the partitions, installing an operating system image that has been
downloaded from the internet, and then installing all the software that is required.  This
\emph{will} require another machine, although the raspberry pi organisations provide imaging software to run on a variety of hardware.
Later stages of the process could be conducted on the raspberry pi, although it will require an additional sd card, so if you have access
to other linux systems it is recommended to use those.
\item Using a skeleton SD Card image, made from the image created in the first option, and then loading up backups that have been made
from the running system
\end{enumerate}

These two options are described in the main section below.

\section{Creating an SD Card from Scratch}
\subsection{Getting The Initial System set up on an SD card}
Working on the `preparation' machine, download a working image of the 64bit Raspberry PI OS (Linux, Debian) System.  At the time of writing
this is a version called `Bulleseye' and is downloadable from
\url{https://www.raspberrypi.com/software/operating-systems/#raspberry-pi-os-64-bit}

This page shows various downloads, but the `Raspberry Pi OS Lite' package is the one required. This starts
with a minimal number of packages installed (more can be added later). Choose the zip file version,
which if you click on the link will load.

When this zip file has finished downloading extract the contents (an `.img' file).
To install this image on the card we can download and install the Raspberry PI Imager on the computer that you have available.

If you do not have a separate linux environment to use, you can use the raspberry pi image we just made to act as you linux environemnt
for the rest of the steps - but you will been a separate sd card and usb adaptor to build the final image following the instructions below.



If you have a linux environment available follow the instructions given here:- 

\begin{lstlisting}
  sudo dd status=progress bs=4M if=imagefilename.img of=/dev/sdX  #X whatever the device holding the sdcard is
\end{lstlisting}

This will create an SD card with two partitions filling up about 1.4GB of the space on the card.
Before the this card can be put in the Raspberry Pi space needs to be allocated for one more
partition.  The easiest way to do that is with  \texttt{fdisk}.  Ultimately the second partition is
going to take the remainder of the SD card minus a 1GB partition at the end used to hold the swap
file\footnote{We are going to be using the `btrfs' filesystem that will not work with file based swap system in the 
stardard image. It is therefore necessary to create an swap partition. This may be a controversial decision as it causes
some diffulty maintaining the initramfs that then becomes necessary when upgrading the kernel.  However, the ability it gives to
make backups so much easier, and the fact that it treats the sdcard better for areas that keep }.  It is necessary to
take the scary step of deleting the second partition and then recreating it at exactly the same
start location, but with the end location 1GB (we try to stay on 4M boundaries in order to line files up
with the erase boundaries of the sd card) from the end of the card. Provided we do recreate this
second partition at exactly the same place as the original, no data will be lost from it.

Follow the following steps:-
\begin{enumerate}
\item With the card still in the /dev/sdX as above, run \texttt{fdisk /dev/sdX}.
\item Press `p' to print the partition list and note down the starting sector of partition 2.  The
list should also give sector number for the last sector on the card. Subtract 2097152\footnote{1GB
in 512B sectors - the 'bc' program is good for this calculation.} from it and round this down so it exactly 
divisible by 8192. This as the start sector
for partition 3. Partition 2 should end one sector before this.
\item Press `d' to delete this partition.
\item Press `n' to recreate this partition.  Make it a primary partiion and use the starting sector
noted in step 2 above and the end sector as the value also noted in Step 2 minus 1. Don't remove the
ext4 signature when prompted.
\item Press `p' to re print the partition list and note the ending sector number for the second partition.
\item Press `n' to add a new partition starting at one sector after the ending sector noted above.
Again it should be a primary partition and should fill the rest of the disk.
\item Press `t' to specify the type of this partition as `82'. which is the type of a swap partition.
\item Press `w' to save the partition table back to the card and exit the program.
\end{enumerate}

We can now turn partition 3 into a swap partiion
\begin{lstlisting}
sudo mkswap -L swap /dev/sdX3
\end{lstlisting}

We should now have an image that will at least run in a Raspberry Pi

\subsubsection{Initial Adjustments}

Put the card into the Raspberry Pi, borrow and connect a keyboard and monitor to it and power it on.  This
should boot up and present a log on prompt. Log on to the ``pi'' account using the password ``raspberry''.

Once logged on the following has to be completed. This can be done with the command \texttt{sudo
raspi-config}:-

\begin{itemize}
\item Update \texttt{raspi-config} to the latest version (Advanced Options/Update)
\item Change the password of the ``pi'' account. I recommend something private to your developments.
\item Set boot options to Console, requiring user to log in.
\item Ensure the international settings are for English(British).
\item Disable all interface options except SSH server, which should be enabled.
\item Give the unit a hostname of ``recorder'' (Advanced Options/Hostname).
\item Set the required GPU memory to 16  (Advanced Options/Memory Split)
\end{itemize}

Exit the \texttt{raspi-config} program and tell the Raspberry Pi to reboot itself.  From this point
on direct connection of the keyboard and monitor is no longer needed\footnote{Unless an error is
made that has to be recovered from.} and they can be disconnected and returned to where ever they
were borrowed from!

Now use SSH to connect to the ``pi'' account.  This will be something like \texttt{ssh pi@pas}\footnote{If
the host computer has been used on other sd cards connected to this same host then it may be necessary to
edit ~/ssh/known\_hosts on the host computer to remove old and invalid host keys}.

The following set of tasks need to be performed

\subsubsection{Check that the entire root partition is filled with a filesystem}

Raspbian seems to do this automatically on first start, but it is worth double checking by
issuing the following command:-

\begin{lstlisting}
sudo resize2fs /dev/mmcblk0p2
\end{lstlisting}

This will either do the expansion or report there is nothing to do.
\subsubsection{Use the swap partition}
We can now get the system to use the swap partition with the following steps:-
\begin{lstlisting}
sudo systemctl stop dphys-swapfile
sudo systemctl disable dphys-swapfile
sudo nano /etc/fstab          #Add the line shown below to the file and remove the comment about no swap because of dphys-swapfile
swapon -a
# if all works
sudo apt remove dphys-swapfile
sudo apt autoremove
sudo rm /var/swap
\end{lstlisting}

The line to add the \texttt{/etc/fstab} should look like this
\begin{lstlisting}
LABEL=swap      none        swap  sw                0    0
\end{lstlisting}


\subsection{Convert to btrfs}
\subsubsection{An Overview of the Process}
Previously the process of creating a btrfs filesystem from the contents of the existing ext4 root filesystem was
automatic using the \texttt{btrfs-convert} utility.  However that is no longer supplied and is not recommended either,
so we have to take a new approach.  So we take this in a series of steps
\begin{enumerate}
  \item prepare an \texttt{initramfs} image that includes the btrfs filesystem
  \item adjust the parameters to enable the new system to correctly boot
  \item move offline so we can manipulate the root filesystem
  \item backup the existing ext4 rootfilesystem, delete it and replace it with a btrfs one
  \item make btrfs subvolumes for root and log
  \item restore the rootfilesystem into the new btrfs subvolumes
  \item replace it in the raspberry pi and boot it.
\end{enumerate}

\subsubsection{Setting up initramfs to include the btrfs filesystem module}

The first step is to install the btrfs and initramfs tools that we need. It is also helpful to
install vim, the text editor that we will use and rsync.

\begin{lstlisting}
  sudo apt install vim btrfs-progs initramfs-tools
\end{lstlisting}

Edit \texttt{/etc/initramfs-tools/modules} to add btrfs as the last line. Then use

\begin{lstlisting}
sudo mkinitramfs -o /boot/initramfs8 $(uname -r)
\end{lstlisting}

To create an initramfs that can be used to start up with btrfs

Edit \texttt{/boot/config.txt} file to have the following line at the top

\begin{lstlisting}
initramfs initramfs8 followkernel
\end{lstlisting}

Edit \texttt{/boot/cmdline.txt} so that it looks like the line below:-

\begin{lstlisting}
  console=tty1 root=LABEL=rootfs rootfstype=btrfs rootflags=subvol=_root fsck.repair=no rootwait
\end{lstlisting}

It is also important that whenever the kernel is upgraded we recreate our \texttt{initramfs8}
file for the newer kernel.

Edit the existing \texttt{/etc/kernel/postinst.d/initramfs-tools} with the
following contents appended to the end.

\begin{lstlisting}
  # Add support for moving the new initrd image to the correct place
  case "${version}" in
    *-v8+)  ;;
    *) exit 0
  esac
  mv /boot/initrd.img-${version} /boot/initramfs8
\end{lstlisting}

Finally edit file \textt{/etc/default/raspberrypi-kernel} to uncomment the line
that starts \textt{#INITRD=Yes}



Shutdown the system and remove the SD card.
\subsubsection{Backup existing ext4 filesystem and replace with new btrfs filesysem}

The following steps need to take place without the SD card as the running kernel. As they are purely
file system changes.  It can be done either on the original host machine where the SD card was
formated in the first place, or on the Raspberry Pi using another SD card made from the original
image and used as a host.

Assume that there is a spare mount point at \texttt{/mnt} in whatever environment is being used and that the
SD card we are converting is in \texttt{/dev/sdX}, then the
following commands should be followed

\begin{lstlisting}
sudo -s   #transition to root mode and stay there
mount /dev/sdX2 /mnt
cd ~
mkdir rootfs-backup
rsync -axAHX /mnt/ rootfs-backup/  #backup root
umount /dev/sdX2

wipefs -a /dev/sdX2
mkfs -t btrfs -L rootfs /dev/sdX2
mount -o subvol=/ /dev/sdX2 /mnt
btrfs subvolume create /mnt/_root
btrfs subvolume create /mnt/_log
btrfs quota enable /mnt/_log              #so we can apply a quota to it
btrfs qgroup limit 2g /mnt/_log           #of 2 Gigs
btrfs subvolume create /mnt/_pm2logs      #to be used when we come to set up pm2
btrfs subvolume create /mnt/_recordings   #we will store recordings in a separate subvolume
rsync -axAHX rootfs-backup/ /mnt/_root/   #restore filesystem
mv /mnt/_root/var/log/*  /mnt/_log/       #separate logs
mkdir -p /mnt/_root/mnt/btrfs_pool        #create a place to mount the whole btrfs filesystem
mkdir -p /mnt/_root/mnt/backup            #create a directory for mounting external backup media in future
vim /mnt/_root/etc/fstab  #set the new contents of /etc/fstab as listed below
umount /mnt
\end{lstlisting}

The \texttt{/etc/fstab} file edited above should look like the following

\begin{lstlisting}
proc            /proc     proc  defaults                          0    0
LABEL=boot      /boot     vfat  defaults                          0    2
LABEL=rootfs    /         btrfs defaults,noatime,ssd,subvol=_root 0    0
LABEL=rootfs    /var/log	btrfs defaults,noatime,ssd,subvol=_log  0    0
LABEL=swap      none      swap  sw                                0    0
\end{lstlisting}

We will set up the other subvolumes later when we have added the directories and
applications that use them

Umount and remove the SD card and place in back in the Raspberry Pi and boot it.  If this has worked and you can ssh into the pi
account, you can also safely delete the rootfs-backup directory created earlier.

\section{Adjusting the system so it ready for the recorder application}

\subsubsection{Set up SSH Keys to avoid having to enter a password on each connection}

It is desirable to be able to connect as the ``pi'' user to perform system maintenance tasks without having
to enter a password every time.  This can be done securely by providing the public keys of users
allowed to connect and perform this function - assuming the accounts already
have private keys set up. The public keys need to be copied to the new pi
account and installed in an \texttt{authorized\_keys} file.

Firstly whilst in the ``pi'' account perform the following commands:-

\begin{lstlisting}
mkdir .ssh
chmod 700 .ssh
\end{lstlisting}

Then back on the machine where all the keys are stored \footnote{The `recorder'
without any extension is the private key for this box and can be used to connect
from this box to elsewhere. This will be used, for instance to connect to a
backup machine and set backups there.}

\begin{lstlisting}
scp xxx.pub recorderexx pi@recorder:.ssh/
\end{lstlisting}

Then ssh back to pi@recorder

\begin{lstlisting}
cd .ssh
mv recorder id_ed25519  #set up the private key for this unit
cat *.pub > authorized_keys #set up all the public keys
\end{lstlisting}

It should now be possible to connect directly to pi@pas without entering a password\footnote{This
assumes that the private keys corresponding the the copied public keys are without pass phrases.
This is generally the case with the keys used just because of the convenience.}.

Finally set up so that the root account can ssh to other machines:-

\begin{lstlisting}
  sudo -i  #takes into the /root directory
  mkdir .ssh
  chmod 700 .ssh
  cp ~pi/.ssh/id_ed25519 .ssh/
\end{lstlisting}

\subsubsection{Add additional software needed to run the system}

Initially upgrade to latest versions of software

\begin{lstlisting}
sudo apt-get update
sudo apt-get upgrade
sudo reboot
\end{lstlisting}


Then install the additional software required.

\begin{lstlisting}
  sudo apt-get install git htop wget nginx rsync tar anacron ffmpeg
\end{lstlisting}

So far, apart from creating the specific host name and using it, and creating
the btrfs subvolume _recordings the entire process has been generic for any
raspberrypi 4.  This is a good time therefore to consider snapshotting the
process in some way to prevent having to go through the entire process again.
This is described in the next section 


\section{Creating and using an pre-prepared image} 
\subsection{An overview of what we aiming to achieve}

We want to store a minimal SD card image and related compressed filesystems such
that with very little effort we can turn the combined whole into a working
environment. However we want to go further than that - in that as we update the
operating system we capture those changes, and they can easily been incorporated
into a newly created SD Card.  In otherwords, in the event of a catastrophic
failure of an existing SD Card, we can recreate the operational system (minus
perhaps some of the history logs and or operational data) vary rapidly.

To that end we have considered the structure to date as follows: -

\begin{itemize}
  \item The root filesystem sits on a single btrfs subvolume, minus the second
  by second data (ie \texttt{/var/log/*} files and the \texttt{pm2}
  logs\footnote(We have yet to install pm2, but its log files will be the output
  from our application we might want for diagostic purposes.  We have already
  prepared but not used a btrfs subvolume for them to reside on))
  \item the /boot file hierachy is also a separate filesytem that we can take a
  regular backup of.
  \item a minimal sdcard image comprising only the boot partition and a highly
  truncated second partition.  This second partition will NOT hold any files -
  but instead we will be minimal in size.  When we come to recreate the sdcard,
  we will copy that image file to an SD card and then used the same trick that
  we used when originally creating the card to expand it to the full size with a
  swap partition at the end of the card.  
\end{itemize}

\subsection{Backing up the latest version of the filesystems}

\subsection{Creating the SD Card Image}












\subsubsection{Create the recorder user}

We are going to have a user ``recorder'' who will own and run the recorder application.  
\subsubsection{Set up backup processes}

We want to create some cron jobs that help maintain the system.  The first of these will be a
filesystem check.  Create a file called \texttt{/etc/cron.monthly/btrfsmaintenance} with the
following content:-

\begin{lstlisting}
#!/bin/sh

btrfs scrub start -Bq /dev/mmcblk0p2

\end{lstlisting}

To sit alongside this file we also need to call \texttt{fstrim}.  However with the buster release of
the operating system we can do this simply by
\begin{lstlisting}
sudo systemctl enable --now fstrim.timer
\end{lstlisting}

Now create a file called \texttt{/etc/cron.weekly/backup} with the following content:-

\begin{lstlisting}
  #!/bin/bash
  mount /dev/mmcblk0p2 -o subvol=/ /mnt/btrfs_pool

  if [ -d /mnt/btrfs_pool/backup-pas-week5 ]; then
    btrfs subvolume delete /mnt/btrfs_pool/backup-pas-week5 2>&1 > /dev/null
  fi
  if [ -d /mnt/btrfs_pool/backup-pas-week4 ]; then
    mv /mnt/btrfs_pool/backup-pas-week4 /mnt/btrfs_pool/backup-pas-week5
  fi
  if [ -d /mnt/btrfs_pool/backup-pas-week3 ]; then
    mv /mnt/btrfs_pool/backup-pas-week3 /mnt/btrfs_pool/backup-pas-week4
  fi
  if [ -d /mnt/btrfs_pool/backup-pas-week2 ]; then
    mv /mnt/btrfs_pool/backup-pas-week2 /mnt/btrfs_pool/backup-pas-week3
  fi
  if [ -d /mnt/btrfs_pool/backup-pas-week1 ]; then
    mv /mnt/btrfs_pool/backup-pas-week1 /mnt/btrfs_pool/backup-pas-week2
  fi

  btrfs subvolume snapshot -r /mnt/btrfs_pool/_pas /mnt/btrfs_pool/backup-pas-week1 > /dev/null
  umount /mnt/btrfs_pool

\end{lstlisting}

and a file called \texttt{/etc/cron.month/backup} with the following content:-

\begin{lstlisting}
  #!/bin/bash
  mount /dev/mmcblk0p2 -o subvol=/ /mnt/btrfs_pool

  if [ -d /mnt/btrfs_pool/backup-rootfs-month3 ]; then
    btrfs subvolume delete -c /mnt/btrfs_pool/backup-rootfs-month3 2>&1 > /dev/null
  fi
  if [ -d /mnt/btrfs_pool/backup-rootfs-month2 ]; then
    mv /mnt/btrfs_pool/backuk-rootfs-month2 /mnt/btrfs_pool/backup-rootfs-month3
  fi
  if [ -d /mnt/btrfs_pool/backup-rootfs-month1 ]; then
    mv /mnt/btrfs_pool/backup-rootfs-month1 /mnt/btrfs_pool/backup-rootfs-month2
  fi
  btrfs subvolume snapshot -r /mnt/btrfs_pool/_root /mnt/btrfs_pool/backup-rootfs-month1 > /dev/null

  if [ -d /mnt/btrfs_pool/backup-pas-month3 ]; then
    btrfs subvolume delete -c /mnt/btrfs_pool/backup-pas-month3  2>&1 > /dev/null
  fi
  if [ -d /mnt/btrfs_pool/backup-pas-month2 ]; then
    mv /mnt/btrfs_pool/backuk-pas-month2 /mnt/btrfs_pool/backup-pas-month3
  fi
  if [ -d /mnt/btrfs_pool/backup-pas-month1 ]; then
    mv /mnt/btrfs_pool/backup-pas-month1 /mnt/btrfs_pool/backup-pas-month2
  fi
  if [ -d /mnt/btrfs_pool/backup-pas-week5 ]; then
    btrfs subvolume snapshot -r /mnt/btrfs_pool/backup-pas-week5 /mnt/btrfs_pool/backup-pas-month1
  fi

  umount /mnt/btrfs_pool

\end{lstlisting}

Make sure all three files have excute priviledges


\subsubsection{Prepare for setting up and running production system}

We need to be able to easily connect via ssh to the pas user.  Set this up with
the following commands:-

\begin{lstlisting}
sudo -s
cd /home/pas
mkdir .ssh
chown pas.pas .ssh
chmod 700 .ssh
cd .ssh
cp /home/pi/.ssh/* .
chown pas.pas *
\end{lstlisting}

Now connect as the ``pas'' user (\texttt{ssh pas@pas}), firstly to check that it works, but also
to perform the next few commands to setup ``nvm'' a node version manager and use it to install the latest
(or a new) version of `node` along with some key `node' based command line tools.

Perform the following commands:-

\begin{lstlisting}
wget -qO- https://raw.githubusercontent.com/creationix/nvm/v0.33.0/install.sh | bash
vim .bashrc #add the following line to the end of the file
export PAS_ENVIRO=clinic
# logout and back in again
nvm install v10.16.3       #use the latest version in which pas has been tested
npm install -g pm2 eslint eslint-config eslint-plugin-html babel-eslint mocha polymer-cli
#logout and login as pi
sudo setcap 'cap_net_bind_service=+ep' /home/pas/.nvm/versions/node/v10.16.3/bin/node  #allow node to bind to low ports
\end{lstlisting}

We now need to clone pasv5 from the master repository, and set it up for production.  ssh in again as the pas user.

\begin{lstlisting}
git clone pas@asgard.hartley-consultants.com:pasv5.git
cd pasv5cd
vim .git/hooks/post-commit  #See contents below
chmod +x .git/hooks/post-commit
ln -s -T .git/hooks/post-commit .git/hooks/post-merge
git checkout production
npm install
polymer build
vi .git/hooks/post-commit #Remove the comment # for the three lines related to oldbuild
pm2 start pas.json
crontab -e  #setup evening task to run every day
pm2 install pm2-logrotate
pm2 set pm2-logrotate:max_size 1M
pm2 set pm2-logrotate:retain 5
pm2 set pm2-logrotate:rotateInterval '15 1 1 1 * *'
pm2 startup #note output from this command and run it as user pi (who has sudo capability)
pm2 save #needed because fails when running as user pi
\end{lstlisting}

The following should be the contents of the post-commit hook created above

\begin{lstlisting}
#!/bin/bash
#
#  For production branch, build the application

branch=$(git branch | sed  -n s/^\*\ //p)
version=$(git describe --tags)

cd "$(git rev-parse --show-cdup)"
npm install

if [ "$branch" == "production" ]; then
    npm install
    # first adjust client code to run from previous version whilst we build the new
#    npm run enviro:old
#    mv build oldbuild
    #build the new
    polymer build
    #swap over to it
    npm run enviro
    #remove old client
#    rm -rf oldbuild
else
    npm run enviro
fi
\end{lstlisting}

The following should be the contents of the crontab edited above.

\begin{lstlisting}
NODE_EXEC = /home/pas/.nvm/versions/node/v10.16.3/bin/node
EVENING = /home/pas/pasv5/server/evening.js
WEBENQ = /home/pas/pasv5/server/getwebenqs.js
MAILTO = alan.chandler@hartley-consultants.com

0 19 * * * $NODE_EXEC $EVENING
*/2 8-18 * * 1-6 $NODE_EXEC $WEBENQ
\end{lstlisting}
\section{Updating Node to a New Version}

\begin{lstlisting}
nvm install <new-version>
nvm alias default <new-version>
\end{lstlisting}
Logon as pi user
\begin{lstlisting}
sudo setcap 'cap_net_bind_service=+ep' /home/pas/.nvm/versions/node/<new-version>/bin/node
\end{lstlisting}
relogon as pas user
\begin{lstlisting}
crontab -e and edit NODE_EXEC line to be /home/pas/.nvm/versions/node/<new-version>/bin/node
npm install -g pm2 eslint eslint-config eslint-plugin-html mocha polymer-cli
pm2 updatePM2
pm2 startup #copy output and run as pi user (who has the sudo capability needed)
pm2 stop all, pm2 delete all, pm2 start pas.json, pm2 save
\end{lstlisting}

\section{Creating a Copy of the SD Card Image}

It is possible to create copies of subvolumes using a pair of commands.  \texttt{btrfs send} can read a subvolume
and output some internal commands to sandard output which can be read on its standard input by \texttt{btrfs receive}.
If running on the same machine the simple unix pipe can be used to connect these two programs together.  Alternatively its
possible to use ssh to pipe the data between machines.  So a possible approach might be to use ssh to copy from the production
system to a desktop machine and then repeat the process from the desktop on to a new sd card.

The paragraphs below leads the reader through this process

\begin{itemize}
  \item Ssh into the production system something like \texttt{ssh pi@pas.accuvision.local}.
  \item make readonly snapshots of all the subvolumes as so:-
  \begin{lstlisting}
    sudo mount -o subvol=/ /dev/mmcblk0p2 /mnt/btrfs_pool
    sudo btrfs subvolume snapshot -r /mnt/btrfs_pool/_root /mnt/btrfs_pool/rootbak
    sudo btrfs subvolume snapshot -r /mnt/btrfs_pool/_log /mnt/btrfs_pool/logbak
    sudo btrfs subvolume snapshot -r /mnt/btrfs_pool/_pas /mnt/btrfs_pool/pasbak
    btrfs subvolume create /mnt/btrfs_pool/bootbak
    rsync -axAHX /boot/ /mnt/btrfs_pool/bootbak/
    btrfs property set -ts /mnt/btrfs_pool/bootbak ro true
  \end{lstlisting}
  \item return to the host system and at some appropriate place to create a subvolume
  \item perform the following
  \begin{lstlisting}
    sudo btrfs subvolume create pas
    ssh pi@pas.accuvision.local "sudo btrfs send /mnt/btrfs_pool/rootbak" | sudo btrfs receive pas/
    ssh pi@pas.accuvision.local "sudo btrfs send /mnt/btrfs_pool/logbak" | sudo btrfs receive pas/
    ssh pi@pas.accuvision.local "sudo btrfs send /mnt/btrfs_pool/pasbak" | sudo btrfs receive pas/
    ssh pi@pas.accuvision.local "sudo btrfs send /mnt/btrfs_pool/bootbak" | sudo btrfs receive pas/
  \end{lstlisting}
  \item return to the production system
  \item perform the following:-
  \begin{lstlisting}
    sudo btrfs subvolume delete -c /mnt/btrfs_pool/rootbak
    sudo btrfs subvolume delete -c /mnt/btrfs_pool/logbak
    sudo btrfs subvolume delete -c /mnt/btrfs_pool/pasbak
    sudo btrfs subvolume delete -c /mnt/btrfs_pool/bootbak
    sudu umount /mnt/btrfs_pool
  \end{lstlisting}
  \item return to host system and place an sd card into a usb adaptor and ports
  \item create partitions on a new 16GB sd card using \texttt{fdisk sdX}.  Ideally, partition boundaries should
  match the write restrictions on the SD cards, which on the SanDisk cards being used for PAS means 4M.  This means the
  partition start sector boundaries (with the previous end sector 1 sector before an the last partition ending at the max) should be (along with partition type):
  \begin{itemize}
    \item Partition 1; 8192 type 0x0c
    \item Partition 2; 540672 type 0x83
    \item Partition 3; 2906064 type 0x82
  \end{itemize}
  \item create filesystems on each partition
  \begin{lstlisting}
    mkfs -t vfat /dev/sdX1
    fatlabel /dev/sdX1 boot
    mkfs -t btrfs -L rootfs /dev/sdX2
    mkswap -L swap /dev/sdX3
  \end{lstlisting}
  \item use \texttt{btrfs send} and \texttt{btrfs receive} to create the correct files on the sd card
  \begin{lstlisting}
    mount /dev/sdX1 /mnt
    rsync -axAHX pas/bootbak/ /mnt/
    umount /dev/sdX1
    mount -o subvol=/ /dev/sdX2 /mnt
    btrfs send pas/rootbak | btrfs receive /mnt/
    mv /mnt/rootbak /mnt/_root
    btrfs send pas/logbak | btrfs receive /mnt/
    mv /mnt/logbak /mnt/_log
    btrfs send pas/pasbak | btrfs receive /mnt/
    mv /mnt/pasbak /mnt/_pas
    umount /mnt
  \end{lstlisting}
  The sd card is now a copy of the live production system and can be carried to whereever it is needed.  It is up to the
  reader to decide whether to keep the subvolumes at pas so the futher sd card copies can be made.
\end{itemize}

\subsection{Update Current System to latest}

\begin{lstlisting}
  sudo apt update
  sudo apt upgrade
  sudo apt full-upgrade
  sudo apt --purge autoremove
  sudo reboot  #ensures we are using the lastest kernel with the initramfs that might have been made
\end{lstlisting}
\subsection{Modify Release Used}
We do an automatic edit to catch all re required files
\begin{lstlisting}
  sudo sed -i /deb/s/stretch/buster/g /etc/apt/sources.list
  sudo sed -i /deb/s/stretch/buster/g /etc/apt/sources.list.d/*.list
\end{lstlisting}

\subsection{Update Package Lists}
Get the lastest package lists imported
\begin{lstlisting}
  sudo apt update
\end{lstlisting}

\subsection{Do the Upgrade}
We are now ready to do the upgrade.
\begin{lstlisting}
  sudo apt upgrade
  sudo apt full-upgrade
\end{lstlisting}

The upgrade process is carried out in two steps: First of all, there is a minimal upgrade to avoid
conflicts and then a complete upgrade. The upgrade may take a while depending on your Internet
connection. A certain amount of attention is required during the update process, as some
configuration files are also updated. Therefore, Raspbian asks several times if you want to replace
certain configuration files with a new one. If you have made changes to individual configuration
files yourself in the past, you should press \textbf{D} (display differences) or \textbf{N} (not replace) depending on
the file, otherwise \textbf{Y} (to replace the old configuration file with a new one).

\subsection{Clean Up Installation}

We remove obsolete/out of date packages
\begin{lstlisting}
  sudo apt --purge autoremove
\end{lstlisting}
Finally reboot to pick up the new system
\begin{lstlisting}
sudo reboot
\end{lstlisting}
\section {The Accuvision Certificate Authority}

An essential requirement for the web server part of PASv5 is the use of a certificate to enable the web server to talk \texttt{https} to
the client machines.  In order to be usable this certificate must be trusted by the browser.  The normal approach for this is that the browser
has a list of public certificate authorities who it trusts to have signed the certificate offered up by the web browser.  An agreement in 2015
means that certicate authorities will only sign certificates for domains that they know the requested has control of.

PASv5 is designed to run inside the clinic on a domain \texttt{accuvision.local} which is not, in the public world, controlled by Accuvision. In
fact the \texttt{.local} part is explicitly reservered for anyone to use how they wish inside their private worlds.  So there is no way a trusted
certificate authority will sign a certificate for \texttt{pas.accuvusion.local}.  The solution to this problem is for Accuvision themselves to set
themselves up as a private certificate authority, sign the certificate for \texttt{pas.accuvusion.local}, and then provide a certificate that describes
themselves to be manually installed in browsers where their certification ability makes sense (i.e. inside the clinic).  PASv5 itself provides a download
link of the certificate authorities certificate.

Previously Alan Chandler acted as the certificate authority and kept all the files associated with that role on his home computer, but the certificate he
set up is expiring at the end of 2020, and the replacement has to survive for a long time - maybe long after he is still providing support.  So the files
and instructions for use have been moved into a directory on the pas server (\texttt{/home/pi/AccuvisionCA}) where they can be used in the longer term.

\end{document}
